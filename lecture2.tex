\section{矩阵分解}
\subsection{奇异值分解}
我们首先引入矩阵的奇异值的概念.
\begin{definition}
设$A\in \mathbb R^{m\times n},r(A)=r$,由$X^T(A^TA)X=(AX)^T(AX)\geq 0$可知$A^TA$(或$AA^T$)是一个半正定矩阵,并且$r(A^TA)=r(AA^T)=r$;定义$A$的奇异值为$A^TA$(或$AA^T$)的$r$个正特征值的算术平方根。
\end{definition}
下面我们给出奇异值分解定理:
\begin{theorem}{\textbf{(奇异值分解/SVD)}}\label{Thm2.1}
设$A\in\mathbb R^{m\times n},r(A)=r$,则存在$m$阶正交矩阵$U$和$n$阶正交矩阵$V$使得$A=UDV^T$,其中$D=\begin{pmatrix} \Lambda & \mathbf 0 \\ \mathbf 0 & \mathbf 0 \end{pmatrix},\Lambda=\mathrm{diag}(\sigma_1,\cdots,\sigma_r)$,$\sigma_1,\cdots,\sigma_r$为$A$的全部奇异值。称$U$(或$V$)的列向量为$A$的左(或右)奇异向量。
\end{theorem}
我们给两个计算奇异值分解的例子:
\begin{example}
\begin{enumerate}
    \item 求$A=\begin{pmatrix} 3 \\ 4 \end{pmatrix}$的一个奇异值分解。 
    \item 求$A=\begin{pmatrix} 1&0&1 \\ 
        0&1&-1 \end{pmatrix}$的一个奇异值分解。
\end{enumerate}
\end{example}
约化的奇异值分解
\begin{definition}
利用分块矩阵乘法可将SVD分解写为$A=\sum\limits_{i=1}^r \sigma_iu_iv_i^T$,这被称为$A$的奇异值分解展开,注意到其中每一项都是一个秩一矩阵;再还原为矩阵可得到约化的SVD分解:$A=U_r\Lambda V_r^T$,其中$U_r,V_r$分别为$U,V$的前$r$列构成,称为列正交矩阵,$\Lambda$同上为全部奇异值所构成的$r$阶对角阵。        
\end{definition}
极分解
\begin{definition}
任一实方阵$A$都可分解为一个半正定矩阵和一个正交矩阵的乘积(或反过来):$A=S_1Q_1$(或者$A=Q_2S_2$)。其中$S_1$(或$S_2$)被$A$唯一决定;如果$A$可逆,则$S_1$(或$S_2$)正定且$Q_1$(或$Q_2$)也被$A$唯一决定。
\end{definition}